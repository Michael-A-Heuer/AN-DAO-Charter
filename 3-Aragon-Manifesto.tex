%!TEX root = network-dao-charter.tex
% SPDX-License-Identifier: CC-BY-SA-4.0

\charterchapter{The Aragon Manifesto: A pledge to fight for freedom}
\label{chap:AragonManifesto}

\textbf{We believe the fate of humanity will be decided at the frontier of technological innovation.} 
We will either see technology lead to a more free, open, and fair society or reinforce a global regime of centralized control, surveillance, and oppression. Our fear is that without a global, conscious, and concerted effort, the outlook is incredibly bleak.

The Internet has opened the doors for universal, cross-border, and non-violent collaborative effort to \textbf{fight for our freedom}.

However, the Internet has also opened the doors for global surveillance and manipulation.

We believe humankind \textbf{should use technology as a liberating tool} to unleash all the goodwill and creativity of our species, rather than as a tool to enslave and take advantage of one another.

Thus, Aragon is a fight for freedom. Aragon empowers freedom by creating liberating tools that leverage decentralized technologies.

\textbf{Decentralized technologies provide users unparalleled power} to transact and interact with a level of security never seen before. Thanks to cryptography and economic incentives, users can now own truly sovereign assets, create fully sovereign entities, and build truly sovereign identities. They solidify freedoms that cannot be taken away, not even by actors with sizeable re- sources. This tectonic shift requires a new method for organizing these sovereign individuals: \textbf{decentralized organizations}.

For the first time in history, thanks to blockchain technology and smart contracts, we can now create fully decentralized organizations, which are truly autonomous and unstoppable.

Decentralized organizations change our relationship with \textbf{governance}: from something that is imposed upon us by others, into something we choose to opt into. 
Where we are equally serving and served, rather than just serving.

Building tools to create and manage decentralized organizations will unleash a cambrian explosion of new governance forms, and the competition among them will raise the bar globally.

It will finally allow us to experiment with governance at the speed of software and learn through the empathy of a collective design approach.

Instead of complaining about how badly incentives are set in the world and how poorly resources are allocated, we will have the power to create systems that better align incentives and distribute resources. This is the enlightenment of the century.

Sovereign individuals will be able to freely express themselves and transact with each other \textbf{without any kind of intermediary} exercising their unjustified power and oppressing them.

This may be one of the most important revolutions humanity has ever faced. We need to be careful about how we go about it. Technology can be a double-edged sword. We need to explicitly set out \textbf{the values that Aragon stands for}, in order for our community to always \textbf{uphold and defend them}.

\textbf{We are committed to building organizational forms that defend self-sovereignty — where a user can always exercise choice, either by participating or exiting.}

In today's world, our governing power over society's common goods is negligible and inaccessible.

Even for exercising small changes over your environment (your city, state or corporation), you might have to go through cumbersome and opaque bureaucratic processes.

This happens because the coordination costs were high in the old world. But now, thanks to the Internet and decentralized governance, we can, and must, engineer systems to empower the people's voice to their fullest potential.

Also, if you dislike those broken governance processes, \textbf{you cannot easily exit from them}. 
Some organizations own your identity (governments), some others own your data (Facebook), and some others even own the path for you to exit (borders), restricting your choice of leaving.

\textbf{We must build organizational structures that allow users and other stakeholders to exit with minimal friction} if they fundamentally disagree with the governance of those structures.

\textbf{We are committed to creating collaboration mechanisms in which violence is not only dis-incentivized, but impossible.}

Choices can only be made freely in the absence of coercion and extortion. 
\textbf{Systems should never use violence} as a means to incentivize or disincentivize human behavior.

With cryptonetworks, we can create environments where others cannot even tell where individuals are located or what they look like. We must make violence in these cryptonetworks impossible by protecting users' privacy with cryptography.

\textbf{We are committed to decentralizing power in order to dismantle unjustified power — which usually springs from centralization.}

Power has a natural tendency to reinforce itself and become corrupt. If power becomes centralized, it doesn't have to answer to anyone but itself. Decentralizing power is essential to minimizing corruption over time.

We must strive to create systems in which a large number of diverse stakeholders have a say, in order for common goods to be responsibly governed by their communities and the relationships between stakeholders based on egalitarian values.

\textbf{We are committed to the creation of long-term value versus short-term profit — which in turn, advances regeneration}.

We want these values to last. 
\textbf{We are in this for the long run.} 
In order to make a lasting impact and disrupt existing power structures, \textbf{we must create systems that provide regenerative economic value to their participants}.

We must honor and encourage the communities that sustain Aragon itself, and we must do so by rewarding those who defend the values outlined in this manifesto.

\textbf{We are committed to a world in which every person can participate in these new organizational structures.}

We seek to use technology to lift people from oppression. To be successful, \textbf{we must keep our products open, understandable, and easy to use} for everyone.
